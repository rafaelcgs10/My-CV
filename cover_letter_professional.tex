%%%%%%%%%%%%%%%%%%%%%%%%%%%%%%%%%%%%%%%%%
% "ModernCV" CV and Cover Letter
% LaTeX Template
% Version 1.1 (9/12/12)
%
% This template has been downloaded from:
% http://www.LaTeXTemplates.com
%
% Original author:
% Xavier Danaux (xdanaux@gmail.com)
%
% License:
% CC BY-NC-SA 3.0 (http://creativecommons.org/licenses/by-nc-sa/3.0/)
%
% Important note:
% This template requires the moderncv.cls and .sty files to be in the same
% directory as this .tex file. These files provide the resume style and themes
% used for structuring the document.
%
%%%%%%%%%%%%%%%%%%%%%%%%%%%%%%%%%%%%%%%%%

%----------------------------------------------------------------------------------------
%	PACKAGES AND OTHER DOCUMENT CONFIGURATIONS
%----------------------------------------------------------------------------------------

\documentclass[11pt,a4paper,sans]{moderncv} 
\moderncvstyle{casual}
% CV theme - options include: 'casual' (default), 'classic', 'oldstyle' and 'banking'
\moderncvcolor{grey}
% CV color - options include: 'blue' (default), 'orange', 'green', 'red', 'purple', 'grey' and 'black'

\usepackage{lipsum} % Used for inserting dummy 'Lorem ipsum' text into the template

\usepackage[scale=0.75]{geometry} % Reduce document margins
%\setlength{\hintscolumnwidth}{3cm} % Uncomment to change the width of the dates column
%\setlength{\makecvtitlenamewidth}{10cm} % For the 'classic' style, uncomment to adjust the width of the space allocated to your name

%----------------------------------------------------------------------------------------
%	NAME AND CONTACT INFORMATION SECTION
%----------------------------------------------------------------------------------------

\firstname{Rafael Castro Gonçalves Silva} % Your first name \familyname{} % Your last name

% All information in this block is optional, comment out any lines you don't need
\title{Professional Cover Letter}
%\phone{(000) 111 1112}
%\fax{(000) 111 1113}
\email{rafaelcgs10@gmail.com}
%\homepage{staff.org.edu/~jsmith}{staff.org.edu/$\sim$jsmith} % The first argument is the url for the clickable link, the second argument is the url displayed in the template - this allows special characters to be displayed such as the tilde in this example
%\extrainfo{additional information}
%\photo[70pt][0.4pt]{pictures/picture} % The first bracket is the picture height, the second is the thickness of the frame around the picture (0pt for no frame)
%\quote{"A witty and playful quotation" - John Smith}

%----------------------------------------------------------------------------------------

\begin{document}
\makecvtitle % Print the CV title
I'm a working computer scientist and a developer.

\section{Things that I like and care, but also things that I don't}
I like challenges in programming! But not \textit{fake challenges}
like understanding frameworks, though this is perhaps one of the most
valued things in the industry nowadays.  I like to reason about the
semantics of the programming language or stuff like ``what is the time
complexity of this function?''  Real challenges in programming usually
require you to be creative and \textit{educate yourself} about the
problem.  There is no silver bullet, you need to learn about the thing
that you are solving, you need to read papers about it. It doesn't matter
which framework you are using if the problem is NP-Hard or if the size of
data is so huge that SQL databases can't handle it.

People should notice that what makes someone a good programmer is to
understand Computing Science foundations.  Yes, tools are important to
be more efficient and so on.  And engineering is about using those
tools and the resources that you have to solve some problem.  But a
tool just \textit{stays} important for some time, eventually some other
new tool appears to take its places.  So in this section, I will only
talk about ideas and values that I like and car about, but not the
tools that I use.  See my CV to know about the tools that I use.

I care about code quality, test coverage, static analysis, and
well-written code. Tests are not the ultimate tool in software
quality, but it is the thing that the industry has widely adopted. I
think we should not forget other styles of quality insurance - I mean,
there is a huge and diverse academic research field about that.  If
you are dealing with a distributed system why not model it in a model
checker? And, please, chose static and strongly typed languages, and
go beyond unit tests.

I think that software development should follow some basic ``practical
software engineering'', like CI/CD, code review, git flow and so on.
Communication in programming is also very important, so we should
always make clear ``why some piece of code exists'', other questions
usually are easy to answer by just reading the thing. In my opinion,
understanding the goal of some function is more important than
understanding how it works. A function should have a single and clear
goal. A single function shouldn't solve everything.

It looks like that the Agile Manifesto is misundestood or people are
forgetting its real meaning. Somehow people are selling ``agile'' as a
coach service. That is bullshit! The Agile Manifesto is about values:
``we value individuals and interactions over processes and tools''.
How can you buy professional values for yourself? Or you understand
and have those values or you don't. Okay, you could pay someone to
teach you about those values, but is your decision to follow it. The
decision of having some values is an individual one. You can't expect
a team to behave ``agile'' because the team leader now says that they
will start to use ``agile'', and they usually do that by enforcing
some ritual (just like a cult).  The decision of being ``agile'' is
individual, not a team decision.  My decision is to be ``agile'', but
don't expect me to like your rituals.

\textit{Nowadays companies have values}. I'm fine with that. But some
companies are going crazy with the ``values thing''. It is the
organizational silver bullet of the moment. ``The employees will be
more motivated and they will produce more''.  Really? How
indoctrination\footnote{That is the correct word for this, but
  companies use the euphemism ``culture''.} makes people more
motivated? At least with me this \textbf{obsession} is annoying and it
unmotivates me.  So I may work in a company with values, but not in a
place where this is systematically repeated to the employees.

\section{How I'm like}
I'm a quick learner and learning is the most important thing for me,
so I wouldn't like to work in a place where decisions are made based
only on using the technologies that the employees already
know. Developers should be eager to learn the best technology for this or that
specific problem. I rather work in a place where I'm continuously learning.

I'm usually very comunicative and I don't fill ashamed of asking people
to teach me about something. I don't pretend to know something that I don't.
I enjoy working with teams that do \textit{real team work} and members
suport each other. 

As you probably already noticed, I'm very critical/argumentative, so
I don't want to work in place that I should just obey 

\end{document}
