%%%%%%%%%%%%%%%%%%%%%%%%%%%%%%%%%%%%%%%%%
% "ModernCV" CV and Cover Letter
% LaTeX Template
% Version 1.1 (9/12/12)
%
% This template has been downloaded from:
% http://www.LaTeXTemplates.com
%
% Original author:
% Xavier Danaux (xdanaux@gmail.com)
%
% License:
% CC BY-NC-SA 3.0 (http://creativecommons.org/licenses/by-nc-sa/3.0/)
%
% Important note:
% This template requires the moderncv.cls and .sty files to be in the same
% directory as this .tex file. These files provide the resume style and themes
% used for structuring the document.
%
%%%%%%%%%%%%%%%%%%%%%%%%%%%%%%%%%%%%%%%%%

%----------------------------------------------------------------------------------------
%	PACKAGES AND OTHER DOCUMENT CONFIGURATIONS
%----------------------------------------------------------------------------------------

\documentclass[11pt,a4paper,sans]{moderncv} 
\moderncvstyle{casual}
% CV theme - options include: 'casual' (default), 'classic', 'oldstyle' and 'banking'
\moderncvcolor{grey}
% CV color - options include: 'blue' (default), 'orange', 'green', 'red', 'purple', 'grey' and 'black'

\usepackage{lipsum} % Used for inserting dummy 'Lorem ipsum' text into the template

\usepackage[scale=0.75]{geometry} % Reduce document margins
%\setlength{\hintscolumnwidth}{3cm} % Uncomment to change the width of the dates column
%\setlength{\makecvtitlenamewidth}{10cm} % For the 'classic' style, uncomment to adjust the width of the space allocated to your name

%----------------------------------------------------------------------------------------
%	NAME AND CONTACT INFORMATION SECTION
%----------------------------------------------------------------------------------------

\firstname{Rafael Castro Gonçalves Silva} % Your first name \familyname{} % Your last name

% All information in this block is optional, comment out any lines you don't need
\title{Professional Cover Letter}
%\phone{(000) 111 1112}
%\fax{(000) 111 1113}
\email{rafaelcgs10@gmail.com}
%\homepage{staff.org.edu/~jsmith}{staff.org.edu/$\sim$jsmith} % The first argument is the url for the clickable link, the second argument is the url displayed in the template - this allows special characters to be displayed such as the tilde in this example
%\extrainfo{additional information}
%\photo[70pt][0.4pt]{pictures/picture} % The first bracket is the picture height, the second is the thickness of the frame around the picture (0pt for no frame)
%\quote{"A witty and playful quotation" - John Smith}

%----------------------------------------------------------------------------------------

\begin{document}
\makecvtitle % Print the CV title
%----------------------------------------------------------------------------------------
%	EDUCATION SECTION
%----------------------------------------------------------------------------------------
I'm a working computer scientist and a developer.

\section{Things that I like and care}
I like challenges in programming! But not \textit{fake challenges}
like understanding frameworks, though it perhaps is one of the most
valued things in the industry nowadays.  I like to reason about the
semantics of the programming language or stuff like ``what is the time
complexity of this function?''  Real chanllenges in programming usualy
requires you to be creative and \textit{educate yourself} about it.
There is no silver bullet, you need to learn about the thing that your
solving, you need to read papers and about it! It doesn't matter which
framework you are using if the problem NP-Hard or if the size of data
is so huge that SQL databases can't hadle it.

People should notice that what makes someone a good programmer is to
undestand Computing Science foundations.  Yes, tools are important
to be more efficient and so on.
And engineering is about using those tools and the
resources that you have to solve some problem.  But a tool just
\textit{stay} important for some time, eventualy some other new tool
appears to take its places.  So in this section I will only talk about ideas
ands values that I like and care, but not the tools that I use.

I care about code quality, test coverage, static analysis, and
well-written code. Tests are not the ultimate tool in software
quality, but it is the thing that the industry has widely adopted. I
think we should not forget other styles of quality insurance - I mean,
there is a huge and diverse academic research field about that.  If
you are dealing with a distributed system why not model it in a model
checker?

I think that software development should follow some basic ``practical
software engineering'', like CI/CD, code review, git flow and so on.
Comunication in programming is also very importat, so we should always
make clear ``why some piece of code exists'', other questions usualy
are easy to answer by just reading the thing. In my opinion, undestanding
the goal of some function is more important than undestanding how it
does things.

\section{How I'm like}
I'm nerd, progressist and I like dogs.

I'm a quick learner and learning is the most important thing for me,
so I wouldn't like to work in a place where decisions are made based
only on using the technologies that the employees already
know. Developers should be eager to learn the best technology for this
specific problem. I rather work in place where I'm continuasly learning,

\end{document}
