%%%%%%%%%%%%%%%%%%%%%%%%%%%%%%%%%%%%%%%%%
% "ModernCV" CV and Cover Letter
% LaTeX Template
% Version 1.1 (9/12/12)
%
% This template has been downloaded from:
% http://www.LaTeXTemplates.com
%
% Original author:
% Xavier Danaux (xdanaux@gmail.com)
%
% License:
% CC BY-NC-SA 3.0 (http://creativecommons.org/licenses/by-nc-sa/3.0/)
%
% Important note:
% This template requires the moderncv.cls and .sty files to be in the same
% directory as this .tex file. These files provide the resume style and themes
% used for structuring the document.
%
%%%%%%%%%%%%%%%%%%%%%%%%%%%%%%%%%%%%%%%%%

%----------------------------------------------------------------------------------------
%	PACKAGES AND OTHER DOCUMENT CONFIGURATIONS
%----------------------------------------------------------------------------------------

\documentclass[11pt,a4paper,sans]{moderncv}
\moderncvstyle{casual}
% CV theme - options include: 'casual' (default), 'classic', 'oldstyle' and 'banking'
\moderncvcolor{grey}
% CV color - options include: 'blue' (default), 'orange', 'green', 'red', 'purple', 'grey' and 'black'

\usepackage{lipsum} % Used for inserting dummy 'Lorem ipsum' text into the template

\usepackage[scale=0.75]{geometry} % Reduce document margins
%\setlength{\hintscolumnwidth}{3cm} % Uncomment to change the width of the dates column
%\setlength{\makecvtitlenamewidth}{10cm} % For the 'classic' style, uncomment to adjust the width of the space allocated to your name

%----------------------------------------------------------------------------------------
%	NAME AND CONTACT INFORMATION SECTION
%----------------------------------------------------------------------------------------

\firstname{Rafael Castro Gonçalves Silva} % Your first name \familyname{} % Your last name
\familyname{} % Your last name

% All information in this block is optional, comment out any lines you don't need
\title{Cover Letter}
%\phone{(000) 111 1112}
%\fax{(000) 111 1113}
\email{rafaelcgs10@gmail.com}
%\homepage{staff.org.edu/~jsmith}{staff.org.edu/$\sim$jsmith} % The first argument is the url for the clickable link, the second argument is the url displayed in the template - this allows special characters to be displayed such as the tilde in this example
%\extrainfo{additional information}
%\photo[70pt][0.4pt]{pictures/picture} % The first bracket is the picture height, the second is the thickness of the frame around the picture (0pt for no frame)
%\quote{"A witty and playful quotation" - John Smith}

%----------------------------------------------------------------------------------------

\begin{document}
\makecvtitle % Print the CV title
I'm a working computer scientist and a developer.

\section{Prelude}
This is a text about myself, so you can know a little bit about me.
I did this text in an informal style of writing, so you can imagine me saying what is written here.
Also, notice that this is being consistently improved and updated

\section{How I'm like}
Usually I describe myself as nerd, progressist and cachorrista (a brazilian neologism for dog person).

Let me be more detailed about those things:

I enjoy nerd stuff like video games, fantasy movies and technology in general.
I know, very typical.
Second, by progressist I mean someone that agrees with progressism: the idea that we should embrace changes that can cause improvements in our life quality (e.g., weed legalization).
And last: I like dogs a lot.

\medskip

Of course, I'm not just three things.

\medskip

Another thing that I'm is playful, and I'm far from the stereotype of a serious adult.
Sure I can have serious conversations, but I just find life better to live if I can joke every now and then.
Though I enjoy making jokes, I don't like it when it offends someone that did nothing to deserve it.
I hate when I see someone being humiliated if the person don't really deserve that.
I can't watch \textit{The Office} because I fell like throwing up when I see Michael humiliating his employees that have done nothing to deserve it.
Search for \textit{The Office — Try My Cookie Cookie} on YouTube and you will know what I'm talking about.

\medskip

More on me:

\medskip

I usually avoid things that cause me stress.
If I have the option to spend money so I won't be bothered, I do it.
Sometimes I just quit the thing that is causing me pain or anguish.
I don't mean that I'm a quitter, I just question myself: do I need this?
And depending on the answer I do the benevolent thing to myself.
For example,  my first job was causing me severe stress because of too much pressure, unrealistic deadlines, toxic culture, so I quitted after just two months working there.
It was there right call, so I could move on to the next thing.

\medskip

I'm eager to learn and learning is quite important to me.
This characteristic of mine made me want to avoid to work in the software industry.
My vision was partially correct I would say:
to a company the best tools are the ones that you can easily find employees to hire for.
But, usually, those tools are not really the best ones for the job and are the ones that I don't want to learn.
I learned Ruby for my job, but don't think it was an interesting thing to learn.
However, I have learned very interesting things in my current job.
So I don't see working in the industry with the same pessimist eyes as before.

\medskip

But my opinion, in general, stays the same: developers should be eager to learn the best technology for this or that specific problem.
I think the problem lives on the fact that programmers most of the time are stuck with the same languages and tools, they should more often look out there and see that languages aren't all the same, and they didn't stop innovating in the 90s.
In conclusion about that, I rather work in a place where I'm continuously learning interesting things.
There are some very interesting new languages like Rust that I wish to see in the industry more often.
However, I understand that you don't see a Rust programmer under every rock like JavaScript programmers.

\medskip

I'm usually very communicative and I don't fill ashamed of asking people to teach me about something.
I don't pretend to know something that I don't.
Not only that, but I enjoy working with teams that do \textit{real team work} with members that support each other.

\section{My life in a nutshell}
I was born in 1993 in the city of São Paulo (Brazil), but when I was three years old I moved to the State of Santa Catarina (SC), in the south of Brazil and I have been living here since then.
I lived in some cities of SC, but at the moment I live in Joinville.
In 2012, I started my undergrad in Computer Science, here in Joinville, at the Santa Catarina State University.
For the first time in my life I noticed that I was doing something that I enjoyed.
I loved the courses of Automata Theory, Compilers, Formal Methods and Computability Theory.
I guess I discovered myself in theoretical computer science.
After my undergrad, I started my masters at this same university.

\medskip

\section{Academic me}
At the end of my undergrad I discovered that programs can be certified in respect to a specification using proofs.
I already knew Curry-Howard correspondence, but when I found proof assistants like Coq and Isabelle, and languages with dependend types like Agda and Idris, I was very surprised to see what people were doing with it.
For example, I found out CompCert, a C compiler with a fully verified backend.
Come on, this is the coolest piece of software ever made!
I started the masters and my advisor had experience with type systems for programming languages, specifically type inference algorithms.
So I decided that my masters would be using a proof assistant and would have something related to type inference.

\medskip

Long story short: I did a Coq certification of algorithm W using Hoare logic using a monad.
Google \textit{Monadic W in Coq} and you will find the paper.

\medskip

After the finishing my masters, I decided to look for a PhD, but things didn't go so far in this matter.
In addition, a pandemic started so this plan is delayed.

\section{Professional me}
I had just finished my masters, I couldn't start the PhD and I had no professional experience.
I decided that the least I could do is to solve the latter.
Looking for a job was quite sad because I don't really see many jobs opportunities that I find interesting.
I managed to find a job that looked quite convenient for logistic reasons and I decided to take it.
As I said, I didn't like it in the first few weeks.
The problem were so many, just to get you a glance: it was a startup following this stupid trend of startups that think they are a family.
More on that later.

\medskip

So I quitted that job and started looking for another one.
I found a place that some colleagues from the university were working at and they said this was a good place to work.
The company is an outsourcing one, they provide their service as teams of programmers.
So, I applied and I was hired.
This company is called Magrathea.
Yes, the planet that makes planets.

\medskip

I was allocated to a team working in project that is quite interesting.
Being brief: it's a distributed system with concurrency scenarios that must process events in whatever order, so it can keep a materialized view correctly updated.
At the moment I'm writing this text I'm still working in this project and the experience so far is great.
I find it very satisfactory to work with a team of people that have a higher level of abstract thinking and can write non-trivial algorithms.
Working with them made me discover how it's like to work with a team that help each other in order to kill a dragon.
Now I can say that I can be a good professional.

\medskip

I and my teammates joke that after this project we don't know what to do because everything else looks quite boring.
So, if you are an employer reading this I think you should understand the kind of developer I'm:
I don't like trivial problems, I need to be constantly challenged, I like mathematics, I don't like to see technology as a mystical thing as many frameworks make it look like.

\medskip

I know I don't have much experience as a developer, but now I at least see the possibility of an interesting professional life for me.
The experience that I had in my undergrad and masters was amazing, and I could say that this kind of experience also matters in the industry.
The abstract thinking that I developed in the academy came in hand in many situations so far.
Though I still wish to do a PhD, I don't see this as an issue for my professional life.
If you do, you missed the chance of stop reading this text long ago.

\section{The ``To whom it may concern'' section}
If your received this cover letter, this section concerns you.

\medskip

Some of the most obvious things that job interviewers ask are:
``why do you want to leave your current job?'', or
``why are looking for a new job?'', and variations.
Those are very relevant questions that I hope to better answer here than I could possibly to in an interview.

\medskip

I have a few reasons. And those reasons are true when I'm writing this (May 2021), but I don't know for how long they will be true.

\medskip

The first one is that I decided to take the leap and take the best job that I can find for myself, but this time I'm considering the entire world instead of just Brazil.
As any good professional, I seek to work in the best company that I can find.
But of course, there are several restrictions to that.
When I took my current job I did that considering the restrictions that I had at the time, but a lot of things changed in my life and most of those restrictions are gone.
Those restrictions were personal things, like leaving my loved ones.

\medskip

Second, right now the market of the technology sector is so hot that I feel like it would be stupid to ignore the opportunities, and not at least consider them.
Many of my friends and colleagues are enjoying this moment to find a better place to work and have a better life.
I've decided to do the same.

\medskip

At last, I am very curious about other cultures and I already wanted to live in other countries and I just realized that now is the time.

\medskip

Considering what I said, I hope that it's clear that I want to make a big change in my life.
And I know this change is something that will take a lot of effort and time.
So, let me be clear that I'm aiming to stay a good few years in my next company, and I'm prepared for that.

\section{Things that I like}
TODO
% I like challenges in programming! But not \textit{fake challenges} like understanding frameworks, though this is perhaps one of the most valued things in the industry nowadays.
%  % I like to reason about the semantics of the programming language or stuff like ``if I do this odd thing in JS, what will be the behavior?'' or ``how do we avoid this concurrency problem?''.
% Real challenges in programming usually require you to be creative and \textit{educate yourself} about the problem.
% There is no silver bullet, you need to learn about the thing that you are solving, you need to read papers about it, you need to read the documentation of the tool that you are using.
% It doesn't matter which framework you are using if the problem is NP-Hard or if the data size is so huge that SQL databases can't handle it.

\section{The last cool thing that I did as a job}
TODO

\section{Some thoughts on what is a good programmer}
TODO
% People should notice that what makes someone a good programmer is to understand Computing Science foundations.
% Yes, tools are important to be more efficient and so on.
% And engineering is about using those tools and the resources that you have to solve some problem.
% But a tool just \textit{stays} important for some time, eventually some other new tool appears to take its places.

\section{Disclaimer on the following rant sections}
The following sections have a strong intonation, meaning that I have a very assertive position on those things and that can be seen as arrogant by some people.
I personally don't usually take offense on assertive opinions, specially if it is about something technical.
I understand that criticism can be taken to the personal level, specially if it is about something loved, but I hope you have the maturity to do not do that on criticism about technology.
On the other hand, this kind criticism can be very constructive.

\section{My thoughts on Software Engineering in practice (Rant Section)}
TODO
% Tests are not the ultimate tool in software
% quality, but it is the thing that the industry has widely adopted.
% Before ranting about tests like Dijkstra, anyone should at least acknowledge
% the different kinds of tests that exists and understand the limitations of each.
% In my experience, integration tests can be extremely helpful in discovering
% connor cases of the behavior of the system. Though, unit tests are just crap
% made necessary by dynamic typed languages. Useful test are about general
% and connor cases behavior, not if the code is glued nicely. Let's make tests
% about behavior, not interface connections.

% It looks like that the Agile Manifesto is misunderstood or people are
% forgetting its real meaning. Somehow people are selling ``agile'' as a
% coach service. That is bullshit! The Agile Manifesto is about values:
% ``we value individuals and interactions over processes and tools''.
% How can you buy professional values for yourself? Or you understand
% and have those values or you don't. Okay, you could pay someone to
% teach you about those values, but is your decision to follow it. The
% decision of having some values is an individual one. You can't expect
% a team to behave ``agile'' because the team leader now says that they
% will start to use ``agile'', and they usually do that by enforcing
% some ritual (just like a cult).  The decision of being ``agile'' is
% individual, not a team decision.  My decision is to be ``agile'', but
% don't expect me to like your rituals.

% I think that software development should follow some basic ``practical
% software engineering'', like CI/CD, code review, git flow and so on.
% Communication in software development is also very important, so we should
% always make clear ``why some piece of code exists'', other questions
% usually are easy to answer by just reading the thing. In my opinion,
% understanding the goal of some function is more important than
% understanding how it works. A function should have a single and clear
% goal. A single function shouldn't solve everything.

% I think we should not forget other styles of quality insurance - I mean, there
% is a huge and diverse academic research field about that. If you are dealing
% with a distributed system why not model it in a model checker? And, please,
% chose static and strongly typed languages, and go beyond unit tests. Nowdays,
% the academy is putting its verified code in airplanes and in generators for
% nuclear power plants. This kind of knowledge shouldn't be restrictedrestricted to a few nerds
% at INRIA. All programmers should at least know what is Formal Methods.

\section{Rage against Object Orientation (Rant Section)}
My opinion on Object Orientation (OO) is in opposition with the common belief.
OO programming doesn't provide more abstraction, or better organization, or elegance or simplicity, it just provides the illusion of those things.
Let's begin stating what OO is:
an object is an encapsulated state which you can only interact with by sending messages, in other words OO is the association between data and functions.
There are big problems with that:
\begin{itemize}
  \item States being passed all around. In the original concept the messages in OO should send copies of states, not references of states.
        But this is not how 99\% of OO programming languages are implemented.
        This is a big problem because it's harder to debug the code and to implement parallelism becomes painful.
        OO is just global variables with extra steps.
  \item OO has nothing to do with better organization. One way to implement OO is through classes, which is basically a way to define types.
        Well, types and module systems allows you to organize your code because you can give names for some specific kinds of data and group them together.
        Nothing to do with OO.
  \item OO doesn't provide better abstraction.
        Abstraction is hiding irrelevant information (one possible definition).
        When people are presented OO they are showed trivial examples like a class representing something real like the customer,
        and the teacher says this is how OO allows you to abstract a concrete thing because your customer class is some sort of model
        of the real life customer which doesn't have the irrelevant information.
        But in practice OO programs very often have classes that doesn't represent anything real, like services, manager, factories and other doers.
        Those classes are there to fill a gap and because OO demands you to have the association between data and functions.
        So at end of the day the promised of abstraction is not fully delivered, and you end up with many words in your program that have nothing to do with the domain in matter.
\end{itemize}

\medskip

So what should I do instead of OO?
Should I use functional programming?
The opposite of OO is non-OO, or what was usually called procedural programming.
We can combine procedural programming with imperative and functional programming, as we can also combine these with OO.
Procedural programming is just better.

\medskip

So, in conclusion I must say that I actually write OO code, but not by my choice.
Most systems out there are objected oriented and I, as a humble employee, have the job of programming in them.
But of course I would like to see more people stop using OO.

\section{Why dynamic typed languages still exists?}
Let me be straight here:
dynamic typed languages should have died last century and we should only see it on legacy projects.
It's like coal power plants, why they still widely used?
Both are obsolete technologies that we should be taking the effort to substitute them, or at least avoid more adoption.

\medskip

I know, there is a - quote and quote - big debate between dynamic and static typing.
But this debate shouldn't exist.
Every single facility that dynamic typing gives, you can also have with static typing if the person implementing it is not lazy.

\begin{itemize}
  \item Oh, duck typing? Meh, universal polymorphism is much better.
  \item Oh, not writing types? Please, are we still living in caves? Haven't you heard of type inference?
  \item Oh, no compilation step? Well, you can tweak the compiler for compilation speed (in development).
        Unless you need to recompile everything, this shouldn't be a problem in development.
        And nowadays we have fast computers that are affordable.
\end{itemize}

Static typing is just a more advanced technology!

Let's not make a debate about it, ok?

\section{Rage against company's bullshits (Rant Section)}
% \textit{Since the early 2000 the comporate world is going crazy with the ``values and culture'' thing}.
% It is the organizational silver bullet that has been around for too long.
% ``The employees will be more motivated, and they will produce more''.
% Really? How?

% \medskip

% Now you see many companies proudly saying that in this company we have the values X, Y and Z.
% And they also proudly say how they hammer those values in the employees heads.
% Indoctrination is the correct word for this, but companies use the euphemism ``culture''.

% \medskip

% Culture is part of what make us human.
% If you have people, you will have culture, is not a choice.
% What companies do is to try to manipulate the employees so they accept a lot of bullshit, like crushing, and be happy with it.
% In this case, the ``culture'' is not organic, is a synthetic thing for people control.

% \medskip

% Okey, fair disclaimer: I know not all companies are like that and having the values itself is not a problem.
% But how much annoying the company is about this is.

% some institucional values can be good: providing time and tools for self improvment, constructive feedback cycles like 1-1
% but that something that the company gives to the employee, not asking

% the values most of the time are good things

% what is a good professional? not following six or five values in the board.
% kidden-garten

% And if think your employees love your company's culture, well my friend, I can ensure some are just pretending.

% Startups..
%
% makes people more
% motivated? At least with me this \textbf{obsession} is annoying and it
% unmotivates me.  So I may work in a company with values, but not in a
% place where this is systematically repeated to the employees.


\end{document}
