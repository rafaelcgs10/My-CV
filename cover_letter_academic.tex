%%%%%%%%%%%%%%%%%%%%%%%%%%%%%%%%%%%%%%%%%
% "ModernCV" CV and Cover Letter
% LaTeX Template
% Version 1.1 (9/12/12)
%
% This template has been downloaded from:
% http://www.LaTeXTemplates.com
%
% Original author:
% Xavier Danaux (xdanaux@gmail.com)
%
% License:
% CC BY-NC-SA 3.0 (http://creativecommons.org/licenses/by-nc-sa/3.0/)
%
% Important note:
% This template requires the moderncv.cls and .sty files to be in the same
% directory as this .tex file. These files provide the resume style and themes
% used for structuring the document.
%
%%%%%%%%%%%%%%%%%%%%%%%%%%%%%%%%%%%%%%%%%

%----------------------------------------------------------------------------------------
%	PACKAGES AND OTHER DOCUMENT CONFIGURATIONS
%----------------------------------------------------------------------------------------

\documentclass[11pt,a4paper,sans]{moderncv}
\moderncvstyle{casual}
% CV theme - options include: 'casual' (default), 'classic', 'oldstyle' and 'banking'
\moderncvcolor{grey}
% CV color - options include: 'blue' (default), 'orange', 'green', 'red', 'purple', 'grey' and 'black'

\usepackage{lipsum} % Used for inserting dummy 'Lorem ipsum' text into the template

\usepackage[scale=0.75]{geometry} % Reduce document margins
%\setlength{\hintscolumnwidth}{3cm} % Uncomment to change the width of the dates column
%\setlength{\makecvtitlenamewidth}{10cm} % For the 'classic' style, uncomment to adjust the width of the space allocated to your name

%----------------------------------------------------------------------------------------
%	NAME AND CONTACT INFORMATION SECTION
%----------------------------------------------------------------------------------------

\firstname{Rafael Castro Gonçalves Silva} % Your first name \familyname{} % Your last name
\familyname{} % Your last name

% All information in this block is optional, comment out any lines you don't need
\title{Cover Letter}
%\phone{(000) 111 1112}
%\fax{(000) 111 1113}
\email{rafaelcgs10@gmail.com}
%\homepage{staff.org.edu/~jsmith}{staff.org.edu/$\sim$jsmith} % The first argument is the url for the clickable link, the second argument is the url displayed in the template - this allows special characters to be displayed such as the tilde in this example
%\extrainfo{additional information}
%\photo[70pt][0.4pt]{pictures/picture} % The first bracket is the picture height, the second is the thickness of the frame around the picture (0pt for no frame)
%\quote{"A witty and playful quotation" - John Smith}

%----------------------------------------------------------------------------------------

\begin{document}
\makecvtitle % Print the CV title
To whom it may concern.

\medskip

I am a working computer scientist and a developer.
I will talk a bit about myself and what my academic interests are.

\medskip
I am pursuing a PhD at the University of Copenhagen since early 2022, and my research is about the formal verification of stream processing systems.

Since my Bachelor studies, I have learned about technologies that improve the trustworthiness of software.
My journey started with Automata Theory, went through robust static type systems, and moved towards proof assistants.

\medskip
As I learned about proof assistants like Coq and Isabelle, and languages with dependend types like Agda and Idris, I became very excited about those technologies, and about what people were doing with them.
Eventually, I decided that I wanted to contribute for development of formally verified software that can actually be used in the real world.
This pursuit started in my masters, where I have used Coq to verify a type inference algorithm, and now continues in my PhD research using Isabelle/HOL.

\medskip
After I finished my masters I ended up working in the software industry as most people do.
I was lucky enough to find a very interesting project to be part of, being brief: it was a stream processing system that had to maintain the eventual consistency of materialized views.
This end-up being a good coincidence for the PhD opportunity that I found later.

\medskip
At the moment, I am seeking this opportunity to learn more about the Isabelle Refinement Framework, and working with Prof. Peter Lammich, as something that can bring new frontiers and perspectives for my PhD work.

\medskip
Best regards, Rafael.

\end{document}
